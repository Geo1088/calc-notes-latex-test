\documentclass{article}
\usepackage[utf8]{inputenc}
\usepackage{amsmath}

\title{Notes on Calculus}
\author{Me}

\begin{document}

\maketitle

\section{Derivatives}
The \textbf{derivative} of a function $f(x)$, usually written $f'(x)$, is a \textbf{function} whose output is the instantaneous rate of change of $f$.

\subsection{Example}
Find the derivative of the function $g(x)=2x^3$.
\begin{align*}
g'(x)
& = \lim_{h \to 0} \frac{g(x+h)-g(x)}{x+h-x} \\
& = \lim_{h \to 0} \frac{2(x+h)^3-2x^3}{h} \\
& =\lim_{h \to 0} \frac{2[(x+h)^3-x^3]}{h} \\
& = 2 \cdot \lim_{h \to 0} \frac{(x+h)^3-x^3}{h} \\
& = 2 \cdot \lim_{h \to 0} \frac{(x^3+3x^2h+3xh^2+h^3)-x^3}{h} \\
& = 2 \cdot \lim_{h \to 0} \frac{3x^2h+3xh^2+h^3}{h} \\
& = 2 \cdot \lim_{h \to 0} 3x^2+3xh+h^2 \\
& = 2 \cdot 3x^2+3x(0)+0^2 \\
& = 2 \cdot 3x^2 \\
& = \boxed{6x^2}
\end{align*}

As an extension: At the point $(2, 16)$, the gradient/slope/instantaneous rate of change of $g$ is $6(2)^2=6\cdot4=\boxed{24}$.

An important note that got written on the board: ``$f'(a)$ is the slope of the line tangent to $f$ at the point $(a, f(a))$.''

\section{Integration}
Yes. This is words.
\end{document}
